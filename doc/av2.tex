\documentclass[17pt]{extarticle}

\usepackage[utf8x]{inputenc}
\usepackage{listings}
\usepackage{color}
\usepackage{graphicx}
\usepackage{float}
\usepackage{url}
\graphicspath{{graphs/}}
\definecolor{dkgreen}{rgb}{0,0.6,0}
\definecolor{gray}{rgb}{0.5,0.5,0.5}
\definecolor{mauve}{rgb}{0.58,0,0.82}

\lstset{
    frame=tb,
    language=R,
    aboveskip=3mm,
    belowskip=3mm,
    showstringspaces=false,
    columns=flexible,
    basicstyle={\small\ttfamily},
    numbers=none,
    numberstyle=\tiny\color{gray},
    keywordstyle=\color{blue},
    commentstyle=\color{dkgreen},
    stringstyle=\color{mauve},
    breaklines=true,
    breakatwhitespace=true,
    tabsize=4,
    basicstyle=\tiny
}

\title{AV2}
\author{Rafael da Silva Albuquerque}
\date{05.2024}

\begin{document}

\begin{figure}[t]
    \centering
    \includegraphics[width=0.5\linewidth]{doc/unifor-logo.png}
    \label{fig:my_label}
\end{figure}
\maketitle

\newpage
\section{Questão 1}
\subsection{A) Faça o diagrama de dispersão dos dados}
\begin{figure}[H]
    \centering
    \includegraphics[width=0.7\linewidth]{out/1.png}
    \label{fig:my_label}
\end{figure}

\subsection{B) Estime a reta de regressão que representa os dados}
\begin{figure}[H]
    \centering
    \includegraphics[width=0.7\linewidth]{out/1.png}
    \label{fig:my_label}
\end{figure}

\subsection{C) Determine o coeficiente de correlação}
-0.9090927

\subsection{D) Você acha que conhecer o volume de tráfego é útil para prever a velocidade média? Justifique.}
Sim. Existe uma relação linear entre as variáveis, então pode-se prever a velocidade média para qualquer volume de tráfego.

\section{Questão 2}
\subsection{A) Faça o diagrama de dispersão dos dados}
\begin{figure}[H]
    \centering
    \includegraphics[width=0.7\linewidth]{out/2.png}
    \label{fig:my_label}
\end{figure}

\subsection{B) Encontre a equação do rendimento em função da temperatura}
\[ rendimento = -3.2788 + 0.4861 * temperatura \]

\subsection{C) Determine o coeficiente de determinação}
\[ 0.9946433 \]

\subsection{D) Qual seria o valor estimado de rendimento para uma temperatura de 155 C°?}
\[ 72.06061 \% \]

\section{Questão 3}
\subsection{A) Faça o diagrama de dispersão dos dados}
\begin{figure}[H]
    \centering
    \includegraphics[width=0.7\linewidth]{out/3.png}
    \label{fig:my_label}
\end{figure}

\subsection{B) Estime a equação de regressão}
\[ IBM = 0.2747 + 0.9498 \]

\subsection{C) Determine o coeficiente de determinação}
\[ 0.4695189 \]

\subsection{D) Determine o coeficiente de correlação}
\[ 0.6852145 \]

\subsection{E) Determine as medidas de posição e dispersão.}
Média \\
\[ SandP = 0.89 \] \\
\[ IBM = 1.12 \] \\
Mediana \\
\[ SandP = 1.2 \] \\
\[ IBM = 1.9 \] \\
Moda \\
\[ SandP = 1.2\] \\
\[ IBM = -0.7 \] \\
Variância \\ 
\[ SandP = 6.189889 \] \\
\[ IBM = 11.89289 \] \\
Desvio Padrão \\ 
\[ SandP = 2.487949 \] \\
\[ IBM = 3.448607 \] \\
Coeficiente de variação \\
\[ SandP = 279.5448 \] \\
\[ IBM = 307.9113 \] \\

\subsection{F) Determine o diagrama Boxplot.}
\begin{figure}[H]
    \centering
    \includegraphics[width=0.5\linewidth]{out/3_box_SandP.png}
    \label{fig:my_label}
\end{figure}
\begin{figure}[H]
    \centering
    \includegraphics[width=0.5\linewidth]{out/3_box_IBM.png}
    \label{fig:my_label}
\end{figure}

\section{Questão 5}
\subsection{A) Faça o diagrama de dispersão dos dados}
\begin{figure}[H]
    \centering
    \includegraphics[width=0.7\linewidth]{out/5.png}
    \label{fig:my_label}
\end{figure}

\subsection{B) Estime a equação de regressão}
\[ Melhor_Supino = 38.0428 + 0.5228 * Peso_Corporal \]

\subsection{C) Determine o coeficiente de determinação}
\[ 0.224823 \]

\subsection{D) Determine o coeficiente de correlação}
\[ 0.474155 \]

\subsection{E) Determine as medidas de posição e dispersão.}
Média \\
\[ Peso_Corporal = 72.605 \]
\[ Melhor_Supino = 76 \]

Mediana \\
\[ Peso_Corporal = 72.95 \]
\[ Melhor_Supino = 72.5 \]

Moda \\
\[ Peso_Corporal = 59.8 \]
\[ Melhor_Supino = 55 \]

Variância \\
\[ Peso_Corporal = 246.9942 \]
\[ Melhor_Supino = 300.2632 \]

Desvio Padrão \\
\[ Peso_Corporal = 15.71605 \]
\[ Melhor_Supino = 17.3281 \]

Coeficiente de variação \\
\[ Peso_Corporal = 21.64596 \]
\[ Melhor_Supino = 22.80014 \]

\subsection{F) Determine o diagrama Boxplot.}
\begin{figure}[H]
    \centering
    \includegraphics[width=0.5\linewidth]{out/5_box_Peso.png}
    \label{fig:my_label}
\end{figure}
\begin{figure}[H]
    \centering
    \includegraphics[width=0.5\linewidth]{out/5_box_Supino.png}
    \label{fig:my_label}
\end{figure}

\newpage
\section{Referências bibliográficas}
\url{https://www.cmor-faculty.rice.edu/~heinken/latex/symbols.pdf} \\
\url{https://r-graph-gallery.com/} \\
\url{https://www.google.com/url?sa=t&source=web&rct=j&opi=89978449&url=https://www.tug.org/FontCatalogue/&ved=2ahUKEwiR6I2H54CFAxU7r5UCHZcgAdYQFnoECCQQAQ&usg=AOvVaw33L7IWJzzL53rkS-4ZK6KJ} \\
\url{https://www.overleaf.com/learn}
\url{https://www.baeldung.com/cs/latex-show-url}

\end{document}
